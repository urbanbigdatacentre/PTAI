\documentclass{article}

\usepackage{arxiv}

\usepackage[utf8]{inputenc} % allow utf-8 input
\usepackage[T1]{fontenc}    % use 8-bit T1 fonts
\usepackage{lmodern}        % https://github.com/rstudio/rticles/issues/343
\usepackage{hyperref}       % hyperlinks
\usepackage{url}            % simple URL typesetting
\usepackage{booktabs}       % professional-quality tables
\usepackage{amsfonts}       % blackboard math symbols
\usepackage{nicefrac}       % compact symbols for 1/2, etc.
\usepackage{microtype}      % microtypography
\usepackage{graphicx}

\title{Great Britain Accessibility Indicators 2023: Data descriptor}

\author{
    J Rafael Verduzco-Torres
   \\
    Urban Big Data Centre \\
    University of Glasgow \\
  Glasgow, G12 8RZ \\
  \texttt{\href{mailto:JoseRafael.Verduzco-Torres@glasgow.ac.uk}{\nolinkurl{JoseRafael.Verduzco-Torres@glasgow.ac.uk}}} \\
   \And
    David P McArthur
   \\
    Urban Big Data Centre \\
    University of Glasgow \\
  Glasgow, G12 8RZ \\
  \texttt{\href{mailto:David.Mcarthur@glasgow.ac.uk}{\nolinkurl{David.Mcarthur@glasgow.ac.uk}}} \\
  }


% tightlist command for lists without linebreak
\providecommand{\tightlist}{%
  \setlength{\itemsep}{0pt}\setlength{\parskip}{0pt}}


% Pandoc citation processing
\newlength{\cslhangindent}
\setlength{\cslhangindent}{1.5em}
\newlength{\csllabelwidth}
\setlength{\csllabelwidth}{3em}
\newlength{\cslentryspacingunit} % times entry-spacing
\setlength{\cslentryspacingunit}{\parskip}
% for Pandoc 2.8 to 2.10.1
\newenvironment{cslreferences}%
  {}%
  {\par}
% For Pandoc 2.11+
\newenvironment{CSLReferences}[2] % #1 hanging-ident, #2 entry spacing
 {% don't indent paragraphs
  \setlength{\parindent}{0pt}
  % turn on hanging indent if param 1 is 1
  \ifodd #1
  \let\oldpar\par
  \def\par{\hangindent=\cslhangindent\oldpar}
  \fi
  % set entry spacing
  \setlength{\parskip}{#2\cslentryspacingunit}
 }%
 {}
\usepackage{calc}
\newcommand{\CSLBlock}[1]{#1\hfill\break}
\newcommand{\CSLLeftMargin}[1]{\parbox[t]{\csllabelwidth}{#1}}
\newcommand{\CSLRightInline}[1]{\parbox[t]{\linewidth - \csllabelwidth}{#1}\break}
\newcommand{\CSLIndent}[1]{\hspace{\cslhangindent}#1}

\usepackage{amsmath}
\usepackage{setspace}
\onehalfspacing
\usepackage{booktabs}
\usepackage{longtable}
\usepackage{array}
\usepackage{multirow}
\usepackage{wrapfig}
\usepackage{float}
\usepackage{colortbl}
\usepackage{pdflscape}
\usepackage{tabu}
\usepackage{threeparttable}
\usepackage{threeparttablex}
\usepackage[normalem]{ulem}
\usepackage{makecell}
\usepackage{xcolor}
\begin{document}
\maketitle


\begin{abstract}
This document describes accessibility indicators for Great Britain 2023
dataset.
\end{abstract}

\keywords{
    Accessibility
   \and
    Transport
   \and
    Active travel
   \and
    Health
   \and
    Employment
  }

\hypertarget{background-summary}{%
\section{Background \& Summary}\label{background-summary}}

Accessibility indicators measure the ease of reaching valuable
destinations (Levinson and Wu 2020). The Great Britain Accessibility
Indicators 2023 (AI23) dataset offers small area indicators to key
services, such as: health, education, employment, and urban centres.
This is an updated and extended version of the Public Transport
Accessibility Indicators for Great Britain 2022 (PTAI22) dataset
described here: \url{https://zenodo.org/records/6759240} (J. Rafael
Verduzco Torres and McArthur 2022).

The products described here represent a snapshot of the first quarter of
2023, while the PTAI22 is an earlier version corresponding to the last
quarter of 2021. The indicators of the AI23, where applicable, are
directly comparable to the previous version. The AI23 is extended in the
following ways compared to the PTAI22:

\begin{itemize}
\tightlist
\item
  In addition to public transport, this version considers active modes
  namely, walk and bicycle.
\item
  Also, accessibility to employment is disaggregated according to the UK
  Standard Industrial Classification of Economic Activities (UK SIC).
\item
  Furthermore, the AI23 dataset includes pharmacies as an additional
  health destination.
\item
  The public transport indicators are not only offered for the morning
  peak, but also at the evening off-peak.
\end{itemize}

In sum, the present dataset encompasses a set of ready-to-use
accessibility indicators to employment, general practices (GPs),
hospitals, pharmacies, primary schools, secondary schools, supermarkets,
main urban centres, and urban sub-centres. These are offered for 42,000
small area units in GB, namely at the lower super output area (LSOA) in
England and Wales, and the data zone (DZ) in Scotland.

Research has used accessibility indicators to study a wide variety of
regional and urban outcomes, such as unemployment rates in the labour
market (Bastiaanssen, Johnson, and Lucas 2022), vaccination uptake in
public health (Chen et al. 2023), and residential prices in the property
market (José Rafael Verduzco Torres 2023a). Similarly, these are gaining
relevance in planning and policy making as input for developing
comprehensive project appraisal (Cavallaro, Bruzzone, and Nocera 2023),
20-minute analyses, and as performance benchmark.

\hypertarget{methods}{%
\section{Methods}\label{methods}}

The methods and sources employed for the present dataset largely follow
those described in the earlier version
(\url{https://zenodo.org/records/6759240} (J. Rafael Verduzco Torres and
McArthur 2022)). The reminding of the paper focuses on the key aspects
or extensions of the present version.

The accessibility indicators, \(A\), are constructured using
location-based measures. These include cumulative, relative cumulative,
and dual or nearest opportunity measures. Location-based measures are
estimated from an origin \(i\) and consider opportunities of type
\(W_k\) at potential destinations \(j\). The cumulative measures are
estimated according to the following equation.

\[
\begin{aligned}
A_{ik} &= \sum_{j=1}^{n} W_{jk} f(t_{ij}) \\
f(t_{ij}) &= \left\{
      \begin{array}{ll}
          1 & \quad \text{if }t_{ij} \leq \bar{t} \text{ (threshold value)} \\
          0 & \quad \text{otherwise}.
      \end{array}
    \right.
\end{aligned}
\]

Here, it is assumed that people deem opportunities or services as
reachable if the modelled travel time between the the origin and
destination, \(t{ij}\), is equal or shorter than given threshold,
\(\bar{t}\). All services beyond this limit are disregarded. Relative
cumulative measures inputs the size of the service weighted by the total
number in the region, i.e.~\(W_{jk} / W_k\). Meanwhile, the dual or
nearest opportunity considers the minimum travel time to a destination
where the size of the service is larger that 0. In other words, these
represent the shortest travel time to the nearest facility of type
\(k\), as illustrated in the equation below.

\[
A_{ik} = \min_{j=1}^{n} \{ t_{ij} : W_{kj} > 0 \}
\]

The measures are computed using the \texttt{AccessUK} package for the
\texttt{R} language v0.0.1-alpha (José Rafael Verduzco Torres 2023b).

\hypertarget{origins}{%
\subsection{Origins}\label{origins}}

The population weighted centroid of each of the 41,729 LSOA/DZ are
considered as the origins in accessibility measures measures. These
correspond to the 2011 Census (version last updated in 21 December 2019
for England and Wales, and 26 March 2021 for Scotland).

\hypertarget{key-services-at-destinations}{%
\subsection{Key services at
destinations}\label{key-services-at-destinations}}

Although the indicators consider the same sources of information to
account for the location of services, the closest version available to
the first quarter of 2023 is used unless specified otherwise. Table 1
shows a summary of the total number of services for the different
versions of the accessibility indicators. These display small negative
or positive variations. The urban centre data employs the same input
data. Thus, the figures remain unchanged.

\begin{table}[!h]

\caption{\label{tab:unnamed-chunk-2}Destionation summary}
\centering
\begin{tabu} to \linewidth {>{\raggedright}X>{\raggedright}X>{\raggedright}X}
\toprule
Destination & Total in 2022
(PTAI22) & Total in 2023
(AI23)\\
\midrule
Employment & 30 067 975 & 30 898 620\\
GPs & 7 887 & 7 756\\
Hospitals & 1 510 & 1 572\\
Pharmacies & NA & 12 983\\
Primary schools & 19 853 & 19 830\\
\addlinespace
Secondary schools & 3 457 & 3 466\\
Supermarkets & 6 478 & 6 392\\
Urban centre: Main & 182 & 182\\
Urban centre: Subcentre & 421 & 421\\
\bottomrule
\end{tabu}
\end{table}

\hypertarget{employment}{%
\subsubsection{Employment}\label{employment}}

In addition to accessibility to all types of employment, the AI23
dataset offers measures disaggregated by broad industrial group
according to the UK SIC (see the following URL for a detailed
description of the classification used:
\url{https://www.ons.gov.uk/methodology/classificationsandstandards/ukstandardindustrialclassificationofeconomicactivities}).
Table 2 presents the UK SIC grouping equivalence with the names used for
the accessibility indicators.

\begin{table}[!h]

\caption{\label{tab:unnamed-chunk-3}Broad industrial groups abbreviation}
\centering
\begin{tabu} to \linewidth {>{\raggedright}X>{\raggedright}X}
\toprule
Inidcator name in the IA23 & SIC broad group classification\\
\midrule
employment\_agriculture\_1 & 1 : Agriculture, forestry \& fishing (A)\\
employment\_mining\_2 & 2 : Mining, quarrying \& utilities (B,D and E)\\
employment\_manufacturing\_3 & 3 : Manufacturing (C)\\
employment\_construction\_4 & 4 : Construction (F)\\
employment\_motor\_5 & 5 : Motor trades (Part G)\\
\addlinespace
employment\_wholesale\_6 & 6 : Wholesale (Part G)\\
employment\_retail\_7 & 7 : Retail (Part G)\\
employment\_transport\_8 & 8 : Transport \& storage (inc postal) (H)\\
employment\_accommodation\_9 & 9 : Accommodation \& food services (I)\\
employment\_information\_10 & 10 : Information \& communication (J)\\
\addlinespace
employment\_financial\_11 & 11 : Financial \& insurance (K)\\
employment\_property\_12 & 12 : Property (L)\\
employment\_professional\_13 & 13 : Professional, scientific \& technical (M)\\
employment\_business\_14 & 14 : Business administration \& support services (N)\\
employment\_public\_15 & 15 : Public administration \& defence (O)\\
\addlinespace
employment\_education\_16 & 16 : Education (P)\\
employment\_health\_17 & 17 : Health (Q)\\
employment\_arts\_18 & 18 : Arts, entertainment, recreation \& other services (R,S,T and U)\\
\bottomrule
\end{tabu}
\end{table}

\hypertarget{pharmacies}{%
\subsubsection{Pharmacies}\label{pharmacies}}

The location of pharmacies was obtained from official public health
records. The data for England comes from the `Consolidated
Pharmaceutical List' corresponding to the 2023-24 quarter 1. This was
manually downloaded from the NHS Data portal
(\url{https://opendata.nhsbsa.net/dataset/consolidated-pharmaceutical-list}).
The location of pharmacies in Scotland was accessed from the Public
Health Scotland platform. The `Dispenser Details January 2023' dataset
was downloaded from the URL:
\url{https://www.opendata.nhs.scot/dataset/dispenser-location-contact-details/resource/f44e6a10-4f1f-4ffd-9205-956944bacf95}.
The information for Wales was available from NHS website. The `Pharmacy
Chains' dataset used corresponds to June 2023 (URL:
\url{https://nwssp.nhs.wales/ourservices/primary-care-services/general-information/data-and-publications/pharmacies-in-wales/}).
These data contain address references including the postcode, which was
matched with the ONS postcode dataset to assign a corresponding LSOA/DZ
code.

\hypertarget{hospitals}{%
\subsubsection{Hospitals}\label{hospitals}}

The sources and selection criterion to account for the location of
hospitals remains unchanged and uses the official updated datasets
except for Wales. In the latter case, the PTAI22 used the list of
addresses available on the Health in Wales website
(\url{http://www.wales.nhs.uk/}). However, this is no longer active.
Thus, we used the locations obtained in January 2022.

\hypertarget{travel-costs}{%
\subsection{Travel costs}\label{travel-costs}}

Travel costs in the accessibility indicators \(t_{ij}\) are represented
by the modelled travel time by public transport, bicycle, and walk. The
AI23 use a series of all-to-all travel time matrices computed from each
LSOA/DZ population weighted centroids using \texttt{R5R} software
(Saraiva et al. 2021) for the \texttt{R} programming language. The main
inputs used are the OpenStreetMap road and pedestrian network, bus time
tables from Bus Open Data Service (BODS)
(\url{https://www.gov.uk/transport/bus-services-routes-and-timetables}),
and train time tables from the Rail Delivery Group
(\url{https://www.raildeliverygroup.com/}). The public transport
indicators are estimated for two times of departure, namely 7 a.m. and 9
p.m. on the 7th of March 2023, considering a three hours time window.
Additional details are offered in a separate data descriptor {[}PENDING
REFERENCE{]}.

\hypertarget{data-records}{%
\section{Data records}\label{data-records}}

The accessibility indicators are offered in a series of \texttt{CSV}
files. These are organised by the type of opportunity or service at the
folder level, and by mode at the file level, as shown in the directory
tree diagram below. The structure of the directories is as following:
\texttt{root/{[}NAME\ OF\ SERVICE{]}/access\_{[}NAME\ OF\ SERVICE{]}\_{[}MODE{]}.csv},
where `pt' stands for public transport in the `MODE' element. To
increase visibility, the diagram omits the dissaggregated employment
measures. However, \protect\hyperlink{inventory}{Appendix} includes a
detailed inventory of all the files in the dataset.

\begin{verbatim}
## ../output/
## +-- employment_all
## |   +-- access_employment_all_bicycle.csv
## |   +-- access_employment_all_pt.csv
## |   \-- access_employment_all_walk.csv
## +-- gp_practices
## |   +-- access_gp_practices_bicycle.csv
## |   +-- access_gp_practices_pt.csv
## |   \-- access_gp_practices_walk.csv
## +-- hospitals
## |   +-- access_hospitals_bicycle.csv
## |   +-- access_hospitals_pt.csv
## |   \-- access_hospitals_walk.csv
## +-- main_bua
## |   +-- access_main_bua_bicycle.csv
## |   +-- access_main_bua_pt.csv
## |   \-- access_main_bua_walk.csv
## +-- pharmacies
## |   +-- access_pharmacies_bicycle.csv
## |   +-- access_pharmacies_pt.csv
## |   \-- access_pharmacies_walk.csv
## +-- primary_schools
## |   +-- access_primary_schools_bicycle.csv
## |   +-- access_primary_schools_pt.csv
## |   \-- access_primary_schools_walk.csv
## +-- secondary_schools
## |   +-- access_secondary_schools_bicycle.csv
## |   +-- access_secondary_schools_pt.csv
## |   \-- access_secondary_schools_walk.csv
## +-- sub_bua
## |   +-- access_sub_bua_bicycle.csv
## |   +-- access_sub_bua_pt.csv
## |   \-- access_sub_bua_walk.csv
## +-- supermarkets
## |   +-- access_supermarkets_bicycle.csv
## |   +-- access_supermarkets_pt.csv
## |   \-- access_supermarkets_walk.csv
## \-- variable_descriptor.csv
\end{verbatim}

The \texttt{root/variable\_descriptor.csv} file details the structure
and contents within each of the files outlined in the diagram. This is
shown in Table 3. These contain the corresponding 2011 Census LSOA/DZ
code in the first column. The `mode' column refers to the form of
transport considered for the indicators. Only for public transport
measures contain the column `time\_of\_day', which refers to one of the
two time of departures estimated, namely `am' or `pm'. The
`accessibility' prefix of the columns refers to cumulative measures.
This is given for eight 15-minute time cuts between 15 and 120 minutes.
The relative measures are identified with the `pct' suffix. The
`nearest\_{[}NAME OF SERVICE{]}' column is the travel time in minutes to
the nearest service of type \(k\). This is not available for employment,
as this is treated as aggregated at destinations from source.

\begin{table}[!h]

\caption{\label{tab:unnamed-chunk-5}Variable descriptor}
\centering
\begin{tabu} to \linewidth {>{\raggedright}X>{\raggedright}X}
\toprule
Variable & Description\\
\midrule
geo\_code & 2011 LSOA/DZ geo-code of origin\\
mode & Mode of transport used for the indicators, values: 'pt' = public transport, 'bicycle', 'walk'\\
time\_of\_day & Time of departure, values: 'am' = 7 a.m., 'pm' = 9 p.m. Available for public transport only.\\
access\_[NAME OF SERVICE]\_15 & Cumulative accessibility: Number of services of type k within 15 minutes\\
access\_[NAME OF SERVICE]\_30 & Cumulative accessibility: Number of services of type k within 30 minutes\\
\addlinespace
access\_[NAME OF SERVICE]\_45 & Cumulative accessibility: Number of services of type k within 45 minutes\\
access\_[NAME OF SERVICE]\_60 & Cumulative accessibility: Number of services of type k within 60 minutes\\
access\_[NAME OF SERVICE]\_75 & Cumulative accessibility: Number of services of type k within 75 minutes\\
access\_[NAME OF SERVICE]\_90 & Cumulative accessibility: Number of services of type k within 90 minutes\\
access\_[NAME OF SERVICE]\_105 & Cumulative accessibility: Number of services of type k within 105 minutes\\
\addlinespace
access\_[NAME OF SERVICE]\_120 & Cumulative accessibility: Number of services of type k within 120 minutes\\
access\_[NAME OF SERVICE]\_15\_pct & Relative cumulative accessibility: Number of services of type k within 15 minutes\\
access\_[NAME OF SERVICE]\_30\_pct & Relative cumulative accessibility: Number of services of type k within 30 minutes\\
access\_[NAME OF SERVICE]\_45\_pct & Relative cumulative accessibility: Number of services of type k within 45 minutes\\
access\_[NAME OF SERVICE]\_60\_pct & Relative cumulative accessibility: Number of services of type k within 60 minutes\\
\addlinespace
access\_[NAME OF SERVICE]\_75\_pct & Relative cumulative accessibility: Number of services of type k within 75 minutes\\
access\_[NAME OF SERVICE]\_90\_pct & Relative cumulative accessibility: Number of services of type k within 90 minutes\\
access\_[NAME OF SERVICE]\_105\_pct & Relative cumulative accessibility: Number of services of type k within 105 minutes\\
access\_[NAME OF SERVICE]\_120\_pct & Relative cumulative accessibility: Number of services of type k within 120 minutes\\
nearest\_[NAME OF SERVICE] & Travel time in minutes to the nearest service of type k\\
\bottomrule
\end{tabu}
\end{table}

\hypertarget{technical-validation}{%
\section{Technical validation}\label{technical-validation}}

\hypertarget{usage-notes}{%
\section{Usage notes}\label{usage-notes}}

\hypertarget{code-availability}{%
\section*{Code availability}\label{code-availability}}
\addcontentsline{toc}{section}{Code availability}

All the code used to generate this data set is openly avaiable in the
following GitHub repository:
\url{https://github.com/urbanbigdatacentre/accessibility_indices23}.

\hypertarget{acknowledgement}{%
\section*{Acknowledgement}\label{acknowledgement}}
\addcontentsline{toc}{section}{Acknowledgement}

This work was made possible by ESRC's on-going support for the Urban Big
Data Centre {[}ES/L011921/1 and ES/S007105/1{]}.

\hypertarget{inventory}{%
\section{Appendix 1. Inventory file}\label{inventory}}

\hypertarget{references}{%
\section*{References}\label{references}}
\addcontentsline{toc}{section}{References}

\hypertarget{refs}{}
\begin{CSLReferences}{1}{0}
\leavevmode\vadjust pre{\hypertarget{ref-Bastiaanssen2022}{}}%
Bastiaanssen, Jeroen, Daniel Johnson, and Karen Lucas. 2022. {``Does
Better Job Accessibility Help People Gain Employment? {The} Role of
Public Transport in {Great Britain}.''} \emph{Urban Studies} 59 (2):
301--22. \url{https://doi.org/10.1177/00420980211012635}.

\leavevmode\vadjust pre{\hypertarget{ref-Cavallaro2023}{}}%
Cavallaro, Federico, Francesco Bruzzone, and Silvio Nocera. 2023.
{``Effects of High-Speed Rail on Regional Accessibility.''}
\emph{Transportation} 50 (5): 1685--1721.
\url{https://doi.org/10.1007/s11116-022-10291-y}.

\leavevmode\vadjust pre{\hypertarget{ref-Chen2023}{}}%
Chen, Huanfa, Yanjia Cao, Lingru Feng, Qunshan Zhao, and J. Rafael
Verduzco Torres. 2023. {``Understanding the Spatial Heterogeneity of
{COVID-19} Vaccination Uptake in {England}.''} \emph{BMC Public Health}
23 (1): 895. \url{https://doi.org/10.1186/s12889-023-15801-w}.

\leavevmode\vadjust pre{\hypertarget{ref-Levinson2020}{}}%
Levinson, David M, and Hao Wu. 2020. {``Towards a General Theory of
Access.''} \emph{Journal of Transport and Land Use} 13 (1, 1): 129--58.
\url{https://doi.org/10.5198/jtlu.2020.1660}.

\leavevmode\vadjust pre{\hypertarget{ref-R-r5r}{}}%
Saraiva, Marcus, Rafael H. M. Pereira, Daniel Herszenhut, Carlos Kaue
Vieira Braga, and Matthew Wigginton Conway. 2021. \emph{R5r: {Rapid}
Realistic Routing with {R5}}. Manual.
\url{https://github.com/ipeaGIT/r5r}.

\leavevmode\vadjust pre{\hypertarget{ref-VerduzcoTorres2022}{}}%
Verduzco Torres, J. Rafael, and David McArthur. 2022. {``Accessibility
{Indicators} for {Great Britain}.''} Preprint. June 27, 2022.
\url{https://doi.org/10.5281/zenodo.6759240}.

\leavevmode\vadjust pre{\hypertarget{ref-VerduzcoTorres2023}{}}%
Verduzco Torres, José Rafael. 2023a. {``Revisiting the Capitalization of
Public Transport Accessibility into Residential Land Value: An Empirical
Analysis Drawing on {Open Science}.''} PhD thesis, {Glasgow}:
{University of Glasgow}. \url{http://theses.gla.ac.uk/id/eprint/83588}.

\leavevmode\vadjust pre{\hypertarget{ref-VerduzcoTorres2023a}{}}%
---------. 2023b. \emph{{AccessUK}} (version v0.0.1-alpha). {Glasgow}:
{University of Glasgow}.
\url{https://github.com/urbanbigdatacentre/AccessUK}.

\end{CSLReferences}

\bibliographystyle{unsrt}
\bibliography{references.bib}


\end{document}
